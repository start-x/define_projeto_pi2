\section{Breve descrição}\label{sec}

Planeja-se fazer de uso de  uma bicicleta ergométrica e óculos \textit{Rift}\footnote{Mais informações sobre os óculos \textit{Rift} podem ser obtidas em  \url{http://www.oculusvr.com/rift/}.}, colocando
um usuário em um ambiente pervasivo inerte a realidade virtual proporcionada pelos óculos \textit{Rift}. O usuário sobe na bicicleta e coloca os óculos. 
O que o usuário verá nos óculos é um ambiente virtual de um parque (a ser projetado pelo grupo)
com subidas, descidas, retas e curvas. A medida que o usuário “pedalar no parque”,
sensores instalados na bicicleta coletarão dados que permitam gerar as seguintes informações:
\begin{itemize}
	\item velocidade;
	\item distancia;
	\item calorias queimadas;
	\item energia gerada;
	\item batimentos cardíacos.
\end{itemize}

Os dados serão apresentados diretamente na imagem passada nos óculos. 

Planeja-se também que a bicicleta apresente um sistema conversor de
energia mecânica das pedaladas em energia elétrica capaz de recarregar o celular do usuário.


\section{Materiais} % (fold)
\label{sec:materiais}

\begin{itemize}%[a)]
	\item Óculos Rift \checkmark
	\item Bicicleta ergométrica \checkmark
	\item Bobina geradora de energia elétrica
	\item Sensor de batimentos cardíacos 
	\item Sensor pra detectar quando o usuário virar o guidão da bicicleta %(temos que dar um jeito de fazer curvas!!)
\end{itemize}


\section{Recursos a serem entregues}

\begin{itemize}
	\item Pelo menos um ambiente virtual de um parque
	\item A bicicleta ergonômica modificada
	\item Informações mostradas para o usuário nos óculos:
	\begin{itemize}
		\item Batimentos cardíacos
		\item Distancia percorrida
		\item Media de Velocidade
		\item Mini-mapa do parque$^*$
		\item Velocidade máxima
		\item Tempo ate recarregar a bateria do celular$^*$
	\end{itemize}
\end{itemize}

\section{Recursos que apresentaremos como trabalho futuro}

\begin{itemize}
	\item Introduzir mais de um usuário no mesmo ambiente para gerar competição
	\item Fazer com que a bicicleta se incline de acordo com a condição do ambiente virtual. Ex(mirabolante): \url{https://www.youtube.com/watch?v=2srJXTwnRy4}
\end{itemize}


\section{Papeis}

\begin{itemize}
	\item \textbf{Gestor:} Charles Oliveira
	\item Responsaveis por cada engenharia:
	\begin{itemize}
		\item 	\textbf{Software:} Luiz Oliveira e Lucas Kanashiro
		\item 	\textbf{Energia:} Priscila Pires
		\item 	\textbf{Eletronica:} José Alberto
		\item 	\textbf{Automotiva:} Tatiana Dias
\end{itemize}
\item \textbf{Revisora:} Camila Ferreira

\end{itemize}