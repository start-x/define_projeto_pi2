\section{Breve descrição}\label{sec}

Planeja-se fazer de uso de  uma bicicleta ergométrica e óculos
\textit{Rift}\footnote{oculusvr.com/rift/}, o usuário será colocado em um 
ambiente pervasivo inerte a realidade virtual, proporcionado pelos óculos 
\textit{Rift}. Ao subir na bicicleta e utilizar o óculos, o mesmo terá acesso 
a um ambiente virtual de um parque (a ser projetado pelo grupo). A medida que 
o usuário “pedalar no parque”, cada sensor instalado na bicicleta coletará 
dados que irão permitir gerar as seguintes as informações:
\begin{itemize}
	\item Velocidade;
	\item Distância percorrida;
	\item Calorias queimadas;
	\item Energia gerada;
	\item Batimentos cardíacos.
\end{itemize}

Os dados serão apresentados diretamente na imagem do óculos. 

Planeja-se também que a bicicleta apresente um sistema conversor de
energia mecânica das pedaladas em energia elétrica capaz de recarregar um aparelho eletrônico do usuário.

O produto final será chamado: {\bf Camelo-X}

\section{Materiais} % (fold)
\label{sec:materiais}

\begin{itemize}%[a)]
	\item Óculos Rift \checkmark
	\item Bicicleta ergométrica \checkmark
	\item Bobina geradora de energia elétrica
	\item Sensor de batimentos cardíacos 
	%\item Sensor pra detectar quando o usuário virar o guidão da bicicleta %(temos que dar um jeito de fazer curvas!!)
\end{itemize}

\section{Recursos a serem entregues}

\begin{itemize}
	\item Ambiente virtual de um parque
	\item Bicicleta ergométrica modificada
	\item Informações apresentadas para o usuário no óculos:
	\begin{itemize}
		\item Batimentos cardíacos
		\item Distância percorrida
		\item Média da Velocidade
		\item Velocidade máxima alcançada
		\item Mini-mapa do parque		
		\item Tempo restante para recarregar a bateria do aparelho eletrônico
	\end{itemize}
\end{itemize}

\section{Recursos que apresentaremos como trabalho futuro}

\begin{itemize}
	\item Introduzir mais de um usuário no mesmo ambiente para gerar competição
	\item Fazer com que a bicicleta se incline de acordo com a condição do ambiente virtual. Ex: \url{https://www.youtube.com/watch?v=2srJXTwnRy4}
\end{itemize}


\section{Papeis}

\begin{itemize}
	\item \textbf{Gestor:} Charles Oliveira
	\item Responsaveis por cada engenharia:
	\begin{itemize}
		\item 	\textbf{Software:} Luiz Oliveira e Lucas Kanashiro
		\item 	\textbf{Energia:} Priscila Pires
		\item 	\textbf{Eletronica:} José Alberto
		\item 	\textbf{Automotiva:} Tatiana Dias
\end{itemize}
\item \textbf{Revisora:} Camila Ferreira

\end{itemize}